\textit{Motivation for virtualized approach (increase the number of potential
user groups without increasing the administrative effort); virtualized
research environments; description of infrastructure for virtualized research environments
(OpenStack, startVM etc) on NEMO }

The computer center of the University of Freiburg provides reasonably scaled research
infrastructures like cloud, storage and especially HPC to cater to the needs of various
scientific communities. To operate systems comprised of more than 1000 inidividual nodes
with a small group significant standardization in hardware and software is necessary. To
address the particular needs of individual scientific communities and research groups the
computer is participating in a state-wide HPC federation and cooperating with other
computer centers within the bwHPC-C5 project. The bwForCluster NEMO is focusing on the
domains of neuroscience, microsystems technology and particle physics, offering its
services to all reasearchers of these fields working at one of the nine Baden-Wuerttemberg
universities. The level of granularity of the software stack provided is not fine enough to
directly support the quite special requirements of world-wide efforts like the ATLAS or CMS
experiments. To utilize the system optimally and open the cluster to as many different
needs as possible without overstretching its own man power novel approaches were
necessary. Thus, the eScience group pooled in it's expertise in cloud operation especially
the use of OpenStack as a cloud platform to provide a more flexible software deployment
on the cluster beside the existing module system.

The state sponsored E-Research-Initiative on virtualized research environments (VREs)
provided the perfect environment to bring together both the infrastructure providers like
the computer center and selected research communities e.g. in the two year ViCE project.
The CMS and ATLAS groups cooperate in this project or as a bwHPC-C5 tiger team to tackle
the particular challenges of VREs like provisioning, setup, scheduling and decommissioning.
A VRE in the context of this paper is a complete software stack as it would get installed on a
compute cluster fitted to ATLAS or CMS groups demand. The resulting filesystem is a
container presented as a single file. From the computer center's perspective this container
is seen as a black box requiring no involvement or efforts. From the researchers
perspective the VRE is an individual (virtual) server where everything from the hardware --
at least to a certain degree like CPU or RAM dimensioning -- up to the operating system,
applications and configurations can be controlled solely by the research groups.
To allow more flexible software environments the standard bare metal operation the NEMO
cluster got extended with a parallel installation of OpenStack components \cite{hpc-symp:2016}. They are getting orchestrated from a special management node which
provides the necessary interfaces to the other services. The NEMO cluster as it's siblings
throughout the state use Adaptive's MOAB as a scheduler of compute jobs. To allow
seemless coexistence of both bare metal and virtualied jobs the scheduler has to handle
both of them. As the scheduler does not look into the VRE virtual machines, it cannot assess
if they are actually active or idle. To solve this, a customized glue component got introduced
(to be described in further detail in section XX). The virtual machines started and stopped bythe OpenStack nova(?) components triggered by ??. 

$<$Michael, hier wollen wir evtl. noch etwas mehr dazu sagen!?$>$
