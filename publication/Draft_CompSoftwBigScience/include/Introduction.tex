This paper presents the concepts and implementation of providing a HPC resource
to ATLAS and CMS users accessing external clusters connected to the World-wide
computing grid (WLCG) with the purpose of running data production as well as
data analysis on the HPC host system.  For this purpose, the HPC cluster NEMO at
the University of Freiburg is deploying an OpenStack instance to handle the
virtual machines.  The schedulers on the NEMO and the external resources are
connected through the ROCED service\cite{ROCED}.

The CMS and ATLAS groups cooperated in this project with the HPC team of the
eScience group of the computer center of University of Freiburg (UFR) to tackle
the particular challenges of VREs like provisioning, setup, scheduling and
decommissioning. A VRE in the context of this paper is a complete software stack
as it would get installed on a compute cluster fitted to ATLAS or CMS groups
demand.

