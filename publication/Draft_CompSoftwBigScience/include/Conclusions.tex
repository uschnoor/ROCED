A system for the dynamic, on-demand provisioning of virtual machines to run jobs
in a high energy physics context on an external, not dedicated resource as
realized at the HPC cluster NEMO at University of Freiburg has been
described. An interface between the schedulers of the
host system and the external system from which requests are sent is needed to
monitor and steer jobs in a scalable way. This is implemented in the \Roced package
which is deployed for the described use-cases.
This approach can be generalized to other platforms and possibly also other
forms of virtualized environments (containers).

The CPU performance and usage of the setup have been analyzed.
The expected performance loss due to the virtualization has been found to be
sufficiently small to be compensated by the added flexibility and other benefits
of this setup.

A possible extension of such a virtualized setup is the provisioning of functionalities
for snapshots and migration of jobs. This would facilitate the efficient integration of
long-running monolithic jobs into HPC clusters.


