%%%%%%%%%%%%%%%%%%%%%%% file template.tex %%%%%%%%%%%%%%%%%%%%%%%%%
%
% This is a general template file for the LaTeX package SVJour3
% for Springer journals.          Springer Heidelberg 2010/09/16
%
% Copy it to a new file with a new name and use it as the basis
% for your article. Delete % signs as needed.
%
% This template includes a few options for different layouts and
% content for various journals. Please consult a previous issue of
% your journal as needed.
%
%%%%%%%%%%%%%%%%%%%%%%%%%%%%%%%%%%%%%%%%%%%%%%%%%%%%%%%%%%%%%%%%%%%
%
% First comes an example EPS file -- just ignore it and
% proceed on the \documentclass line
% your LaTeX will extract the file if required
\begin{filecontents*}{example.eps}
%!PS-Adobe-3.0 EPSF-3.0
%%BoundingBox: 19 19 221 221
%%CreationDate: Mon Sep 29 1997
%%Creator: programmed by hand (JK)
%%EndComments
gsave
newpath
  20 20 moveto
  20 220 lineto
  220 220 lineto
  220 20 lineto
closepath
2 setlinewidth
gsave
  .4 setgray fill
grestore
stroke
grestore
\end{filecontents*}
%
\RequirePackage{fix-cm}
%
%\documentclass{svjour3}                     % onecolumn (standard format)
%\documentclass[smallcondensed]{svjour3}     % onecolumn (ditto)
%\documentclass[smallextended]{svjour3}       % onecolumn (second format)
\documentclass[twocolumn]{svjour3}          % twocolumn
%
\smartqed  % flush right qed marks, e.g. at end of proof
%
\usepackage{graphicx}
%
% \usepackage{mathptmx}      % use Times fonts if available on your TeX system
%
\usepackage{url}
% insert here the call for the packages your document requires
%\usepackage{latexsym}
% etc.
%
% please place your own definitions here and don't use \def but
% \newcommand{}{}
%
% Insert the name of "your journal" with
% \journalname{myjournal}
%
\begin{document}

\title{Virtualization of Particle Physics environments on
  High-Performance Computing cluster%\thanks{Grants or other notes
%about the article that should go on the front page should be
%placed here. General acknowledgments should be placed at the end of the article.}
}
%\subtitle{Do you have a subtitle?\\ If so, write it here}

%\titlerunning{Short form of title}        % if too long for running head

\author{Felix B\"uhrer \and Anton Gamel  \and Michael Janczyk \and
  Ulrike Schnoor \and Bernd Wiebelt \and People from KIT
  % etc.
}

%\authorrunning{Short form of author list} % if too long for running head

\institute{U. Schnoor \at
              CERN \\
              \email{ulrike.schnoor@cern.ch}           %  \\
%             \emph{Present address:} of F. Author  %  if needed
           \and
           F. B\"uhrer \at
              Universit\"at Freiburg
}

\date{Received: date / Accepted: date}
% The correct dates will be entered by the editor


\maketitle

\begin{abstract}
Particle Physics experiments at the Large Hadron Collider (LHC) need a high
amount of computing resources for data processing, simulation, and analysis.
High-Performance Computing (HPC) resources provided by universities
can be useful supplements to the existing WLCG computing resources
allocated by the collaboration. At the university of Freiburg, the
shared HPC cluster "NEMO" has been made available to ATLAS and CMS
users accessing NEMO from external collaboration-specific resources.
 To this effect, the full environment corresponding to a WLCG center
 is provided. The interplay between the NEMO and the external
 resources' schedulers is ensured through the ROCED service.
An OpenStack infrastructure is deployed at NEMO to orchestrate the
parallel usage for bare metal and virtualized jobs.
Through the setup, resources are provided to users in an automatized,
on-demand way. The performance of the virtualized environment has been
evaluated for particle physics jobs.

%Insert your abstract here. Include keywords, PACS and mathematical
%subject classification numbers as needed.
\keywords{Virtualization \and Particle Physics \and Grid Computing \and More keywords}
% \PACS{PACS code1 \and PACS code2 \and more}
% \subclass{MSC code1 \and MSC code2 \and more}
\end{abstract}




\section{Introduction}
\label{intro}

This paper presents the concepts and implementation of providing a HPC
resource to ATLAS and CMS users accessing external clusters connected
to the World-wide computing grid (WLCG) with the purpose of running
data production as well as data analysis on the HPC host system.
For this purpose, the HPC cluster NEMO at the University of Freiburg
is deploying an OpenStack instance to handle the virtual machines.
The schedulers on the NEMO and the external resources are connected
through the ROCED service\cite{ROCED}.


%virtualizing the ATLAS software environment to run both data analysis and production on the HPC host system which is connected to the existing Tier-3 infrastructure. Main challenges include the integration into the NEMO and Tier-3 schedulers in a dynamic, on-demand way, the scalability of the OpenStack infrastructure, as well as the automatic generation of a fully functional virtual machine image providing access to the local user environment, the dCache storage element and the parallel file system. The performance in the virtualized environment is evaluated for typical High-Energy Physics applications.
%

\section{Virtualization infrastructure}
\label{sec:openstack}
Description of infrastructure for virtualized research environments
(OpenStack, startVM etc) on NEMO -- Rechenzentrum


\section{Generation of the image}



\subsection{Packer combined with Puppet}
packer\cite{packer}, puppet\cite{puppet}
ATLAS Freiburg
\subsection{Karlsruhe's method}
CMS Karlsruhe

\section{Interfacing the schedulers using ROCED}

\subsection{ROCED}
 CMS Karlsruhe
\subsection{Using HTCondor as front-end scheduler}
Karlsruhe

\subsection{Using SLURM as front-end scheduler}
ATLAS-Freiburg
\section{Performance}



\subsection{HEPSPEC benchmarks}
ATLAS Freiburg
\begin{figure}[htbp]
%% For example, with the graphicx package use
  \includegraphics[width=\columnwidth]{figures/HEPSPECpCPUvsCPU.pdf}
\caption{Results of HEPSPEC benchmark tests: HEPSPEC per number of
  CPUs in dependence of number of CPUs.}
\label{fig:HEPSPECpCPUvsCPU-atlas}
\end{figure}




\subsection{Simulation data production}
ATLAS Freiburg
\subsection{Data analysis}

\subsection{Other performance analysis? Karlsruhe}

\section{Conclusions and Outlook}

















%\subsection{Subsection title}
%\label{sec:2}
%as required. Don't forget to give each section
%and subsection a unique label (see Sect.~\ref{sec:1}).

%\paragraph{Paragraph headings} Use paragraph headings as needed.
%\begin{equation}
%a^2+b^2=c^2
%\end{equation}
%
%% For one-column wide figures use
%\begin{figure}
%% Use the relevant command to insert your figure file.
%% For example, with the graphicx package use
%  \includegraphics{example.eps}
%% figure caption is below the figure
%\caption{Please write your figure caption here}
%\label{fig:1}       % Give a unique label
%\end{figure}
%%
%% For two-column wide figures use
%\begin{figure*}
%% Use the relevant command to insert your figure file.
%% For example, with the graphicx package use
%  \includegraphics[width=0.75\textwidth]{example.eps}
%% figure caption is below the figure
%\caption{Please write your figure caption here}
%\label{fig:2}       % Give a unique label
%\end{figure*}
%
%% For tables use
%\begin{table}
%% table caption is above the table
%\caption{Please write your table caption here}
%\label{tab:1}       % Give a unique label
%% For LaTeX tables use
%\begin{tabular}{lll}
%\hline\noalign{\smallskip}
%first & second & third  \\
%\noalign{\smallskip}\hline\noalign{\smallskip}
%number & number & number \\
%number & number & number \\
%\noalign{\smallskip}\hline
%\end{tabular}
%\end{table}


%\begin{acknowledgements}
%If you'd like to thank anyone, place your comments here
%and remove the percent signs.
%\end{acknowledgements}

% BibTeX users please use one of
%\bibliographystyle{spbasic}      % basic style, author-year citations
%\bibliographystyle{spmpsci}      % mathematics and physical sciences
%\bibliographystyle{spphys}       % APS-like style for physics
%\bibliography{}   % name your BibTeX data base

% Non-BibTeX users please use
\begin{thebibliography}{}
%
% and use \bibitem to create references. Consult the Instructions
% for authors for reference list style.
%
\bibitem{Openstack}
OpenStack Open Source Cloud Computing Software
\url{https://www.openstack.org/}, accessed 2017-01-10

\bibitem{ROCED}
Authors, Article Title, Journal, Volume, , page numbers (year)
=reference for ROCED
\bibitem{packer}

Packer: tool for creating machine and container images for multiple platforms from a single source configuration. 
\url{https://www.packer.io/}, accessed 2017-01-13

\bibitem{puppet}

Puppet Enterprise. ``IT automation for cloud, security, and DevOps.''
\url{https://puppet.com/}, accessed 2017-01-10

%\bibitem{RefJ}
% Format for Journal Reference
%Author, Article title, Journal, Volume, page numbers (year)
% Format for books
%\bibitem{RefB}
%Author, Book title, page numbers. Publisher, place (year)
% etc
\end{thebibliography}

\end{document}
% end of file template.tex

