%%%%%%%%%%%%%%%%%%%%%%% file template.tex %%%%%%%%%%%%%%%%%%%%%%%%%
%
% This is a general template file for the LaTeX package SVJour3
% for Springer journals.          Springer Heidelberg 2010/09/16
%
% Copy it to a new file with a new name and use it as the basis
% for your article. Delete % signs as needed.
%
% This template includes a few options for different layouts and
% content for various journals. Please consult a previous issue of
% your journal as needed.
%
%%%%%%%%%%%%%%%%%%%%%%%%%%%%%%%%%%%%%%%%%%%%%%%%%%%%%%%%%%%%%%%%%%%
%
% First comes an example EPS file -- just ignore it and
% proceed on the \documentclass line
% your LaTeX will extract the file if required
\begin{filecontents*}{example.eps}
%!PS-Adobe-3.0 EPSF-3.0
%%BoundingBox: 19 19 221 221
%%CreationDate: Mon Sep 29 1997
%%Creator: programmed by hand (JK)
%%EndComments
gsave
newpath
  20 20 moveto
  20 220 lineto
  220 220 lineto
  220 20 lineto
closepath
2 setlinewidth
gsave
  .4 setgray fill
grestore
stroke
grestore
\end{filecontents*}
%
\RequirePackage{fix-cm}
%
%\documentclass{svjour3}                     % onecolumn (standard format)
%\documentclass[smallcondensed]{svjour3}     % onecolumn (ditto)
%\documentclass[smallextended]{svjour3}       % onecolumn (second format)
\documentclass[twocolumn]{svjour3}          % twocolumn
%
\smartqed  % flush right qed marks, e.g. at end of proof
%
\usepackage{graphicx}
%
% \usepackage{mathptmx}      % use Times fonts if available on your TeX system
%
\usepackage{url}
% insert here the call for the packages your document requires
%\usepackage{latexsym}
% etc.
%
% please place your own definitions here and don't use \def but
% \newcommand{}{}
%
% Insert the name of "your journal" with
% \journalname{myjournal}
%
\begin{document}

\title{Dynamic Virtualized Deployment of Particle Physics Environments on a
  High-Performance Computing Cluster%\thanks{Grants or other notes
%about the article that should go on the front page should be
%placed here. General acknowledgments should be placed at the end of the article.}
}
%\subtitle{Do you have a subtitle?\\ If so, write it here}

%\titlerunning{Short form of title}        % if too long for running head

\author{Felix B\"uhrer \and Anton Gamel  \and Michael Janczyk \and
  Markus Schumacher \and
  Ulrike Schnoor \and Bernd Wiebelt \and People from KIT
  % etc.
}

%\authorrunning{Short form of author list} % if too long for running head

\institute{U. Schnoor \at
              CERN \\
              \email{ulrike.schnoor@cern.ch}           %  \\
%             \emph{Present address:} of F. Author  %  if needed
           \and
           F. B\"uhrer \at
              Universit\"at Freiburg
}

\date{Received: date / Accepted: date}
% The correct dates will be entered by the editor


\maketitle

\begin{abstract}
Particle Physics experiments at the Large Hadron Collider (LHC) need a high
amount of computing resources for data processing, simulation, and analysis.
High-Performance Computing (HPC) resources provided by universities
can be useful supplements to the existing World-wide LHC Computing Grid resources
allocated by the collaboration. At the university of Frei\-burg, the
shared HPC cluster "NEMO" has been made available to ATLAS and CMS
users accessing NEMO from external collaboration-specific resources.
 To this effect, the full environment corresponding to a WLCG center
 is provided. The interplay between the NEMO and the external
 resources' schedulers is ensured through the ROCED service.
An OpenStack infrastructure is deployed at NEMO to orchestrate the
simultaneous usage for bare metal and virtualized jobs.
Through the setup, resources are provided to users in an automatized,
on-demand way. The performance of the virtualized environment has been
evaluated for particle physics jobs.

%Insert your abstract here. Include keywords, PACS and mathematical
%subject classification numbers as needed.
\keywords{Virtualization \and Particle Physics \and Grid Computing \and More keywords}
% \PACS{PACS code1 \and PACS code2 \and more}
% \subclass{MSC code1 \and MSC code2 \and more}
\end{abstract}




\section{Introduction}
\label{intro}
This paper presents the concepts and implementation of providing a HPC resource
to ATLAS and CMS users accessing external clusters connected to the World-wide
computing grid (WLCG) with the purpose of running data production as well as
data analysis on the HPC host system.  For this purpose, the HPC cluster NEMO at
the University of Freiburg is deploying an OpenStack instance to handle the
virtual machines.  The schedulers on the NEMO and the external resources are
connected through the ROCED service\cite{ROCED}.

The CMS and ATLAS groups cooperated in this project with the HPC team of the
eScience group of the computer center of University of Freiburg (UFR) to tackle
the particular challenges of VREs like provisioning, setup, scheduling and
decommissioning. A VRE in the context of this paper is a complete software stack
as it would get installed on a compute cluster fitted to ATLAS or CMS groups
demand.



\section{Virtualization infrastructure}
\label{sec:openstack}
$\to$Rechenzentrum\\
Motivation for virtualized approach (increase the number of potential
user groups without increasing the administrative effort); virtualized
research environments; description of infrastructure for virtualized research environments
(OpenStack, startVM etc) on NEMO 


\section{Generation of the image}

The image for the virtual machine should be provided in an automatized
way allowing versioning and archiving of the environments captured in
the images. The approaches used in the
different groups are described in the following.

\subsection{Packer combined with Puppet}
$\to$ATLAS Freiburg\\

One approach to generating the image is the open-source tool
\texttt{packer}~\cite{packer}, interfaced to \texttt{puppet}~\cite{puppet}.
\texttt{Packer} allows to configure an image based on an \texttt{.iso} file using a kickstart~\cite{kickstart} file and flexible script-based configuration. 
It also provides an interface to \texttt{puppet} making it a convenient tool if an existing \texttt{puppet} role is to be used for the images. If the roles are defined according to the hostname of the machine as is conventional in \texttt{puppet} with \texttt{hieradata}, the hostname needs to be set in the scripts supplied to \texttt{packer}. Propagation of certificates require an initial manual start of a machine with the same hostname to allow handshake signing of the certificate from the \texttt{puppet} server.

Apart from these initial adjustments, \texttt{packer}'s interface to \texttt{puppet} allows a fully automated image generation with up-to-date configuration.




\subsection{Karlsruhe's method}
$\to$CMS Karlsruhe\\
Another option to employ a fully-automated procedure is to use the OZ toolkit~\cite{OZ}. All requirements and configuration options of an image can be specified via an XML file, called template. The partitioning and installation process of the operating system is fully automated, as OZ will use the remote-control capabilities of the local hypervisor. After the installation of the operating system, additional libraries and configuration files can be installed. Once the image has been built, it is automatically compressed and uploaded to a remote cloud site.
Using this technique allows to build images in a reproducible fashion, as all templated files are version controlled using git. Furthermore, existing template files are easy to adapt to new sites and experiment configurations.

\section{Interfacing the schedulers using ROCED}
Introduction about the requirements for an in-between layer
\\$\to $ Karlsruhe
\subsection{ROCED}
$\to$ CMS Karlsruhe
\subsection{Using HTCondor as front-end scheduler}\label{sec:ROCED:HTCondor}
$\to$ CMS Karlsruhe

\subsection{Using SLURM as front-end scheduler}
$\to$ ATLAS Freiburg\\

Alternatively to the approach described above, the
open-source workload managing system \Slurm~\cite{Slurm} has been interfaced into \Roced by
the ATLAS group at University of Freiburg.
\Slurm provides a built-in functionality for the dynamic
startup of resources in the \textit{Slurm Elastic Computing}
module~\cite{SlurmElastic}.
However, this module has been found to be unsuitable for resources which are not
expected to be available within a fixed time period, in this case due to
the presence of a queue in the host system which may postpone the start
of a resource by a significant, varying period.
In addition the transfer of information, such as error states, from one scheduler to the
other, and therefore to the user, is very limited.
Therefore, \Roced has been chosen as the interface between the
\Moab scheduler on the host system and the \Slurm
scheduler on the submission side.


\begin{figure}

\includegraphics[width=0.95\linewidth]{figures/virtualisierung_ROCED.png}
\caption{Implementation of \Roced with \Slurm on the BFG cluster used by ATLAS researchers.}
\label{fig:slurmRocedBFG}
\end{figure}

The scheduling system is illustrated in Fig.~\ref{fig:slurmRocedBFG}.
For \Slurm, it is necessary that each potential virtual
machine is registered in the configuration at the time of start of the
\Slurm server as well as the client. \Slurm configurations also
need to be in agreement between server and client.
Therefore, a range of hostnames is registered in the configuration in
a way that is mapped to potential IP addresses of virtual machines.
These virtual machines have a fixed number of CPUs and memory assigned and are
registered under a certain \Slurm partition.
When a job is submitted to this partition and no other resource is
available, information from the \Slurm \texttt{squeue} and
\texttt{sinfo} commands is requested and parsed for the required information.

Since the ATLAS Freiburg group comprises three sub-groups, each mapped
to a different production account on \NEMO, special care is taken to
avoid interference of resources used by another account to ensure fair share on \NEMO, while
allowing jobs from one group to occupy otherwise idle resources of another group.

\Roced determines the amount of virtual machines to be started and sends the
corresponding VRE job submission commands to \Moab.
After the virtual machine has booted, the hostname is set to the IP
dependent name which is known to the \Slurm configuration.
A \texttt{cron} job executes several sanity
checks on the system.
Upon successful execution of these tests, the \Slurm client
running in the VM starts accepting the queued jobs.
After completion of the jobs and a certain period of receiving no new jobs from the queue, the
\Slurm client in the machine drains itself and the machine
shuts itself down.
The IP address as well as the corresponding hostname in \Slurm
are released and can be reused by future VREs.



\section{Analysis of performance and usage}

The approach described so far has been implemented and put into
production in the research groups at University of Freiburg
(Physikalisches Institut) and Karlsruhe (... Institut).
The following chapter presents statistical results of the analysis of
the performance of the virtualized setup both in terms of job
performance as well as usage statistics.


\subsection{HEPSPEC benchmarks}
$\to$ ATLAS Freiburg
\begin{figure}[htbp]
%% For example, with the graphicx package use
  \includegraphics[width=\columnwidth]{figures/HEPSPECpCPUvsCPU.pdf}
\caption{Results of HEPSPEC benchmark tests: HEPSPEC per number of
  CPUs in dependence of number of CPUs.}
\label{fig:HEPSPECpCPUvsCPU-atlas}
\end{figure}




\subsection{Production of simulation data}
$\to$ATLAS Freiburg
\subsection{Data analysis}
$\to$ATLAS Freiburg 

\subsection{Usage statistics}
$\to$CMS Karlsruhe?

\section{Conclusions and Outlook}


A system for the dynamic, on-demand provisioning of virtual machines
to run jobs in a High-Energy Physics context on an external, not
dedicated resource as realized at the HPC
cluster ``NEMO'' at University of Freiburg has been described. 
Reasons for the need for an interface between the schedulers of the host system
and the external system from which requests are sent have been
explained. 
The performance and usage have been analyzed for two groups. 

This approach can be generalized to other platforms and possibly also
other forms of virtualized environments (containers).
















%\subsection{Subsection title}
%\label{sec:2}
%as required. Don't forget to give each section
%and subsection a unique label (see Sect.~\ref{sec:1}).

%\paragraph{Paragraph headings} Use paragraph headings as needed.
%\begin{equation}
%a^2+b^2=c^2
%\end{equation}
%
%% For one-column wide figures use
%\begin{figure}
%% Use the relevant command to insert your figure file.
%% For example, with the graphicx package use
%  \includegraphics{example.eps}
%% figure caption is below the figure
%\caption{Please write your figure caption here}
%\label{fig:1}       % Give a unique label
%\end{figure}
%%
%% For two-column wide figures use
%\begin{figure*}
%% Use the relevant command to insert your figure file.
%% For example, with the graphicx package use
%  \includegraphics[width=0.75\textwidth]{example.eps}
%% figure caption is below the figure
%\caption{Please write your figure caption here}
%\label{fig:2}       % Give a unique label
%\end{figure*}
%
%% For tables use
%\begin{table}
%% table caption is above the table
%\caption{Please write your table caption here}
%\label{tab:1}       % Give a unique label
%% For LaTeX tables use
%\begin{tabular}{lll}
%\hline\noalign{\smallskip}
%first & second & third  \\
%\noalign{\smallskip}\hline\noalign{\smallskip}
%number & number & number \\
%number & number & number \\
%\noalign{\smallskip}\hline
%\end{tabular}
%\end{table}


%\begin{acknowledgements}
%If you'd like to thank anyone, place your comments here
%and remove the percent signs.
%\end{acknowledgements}

% BibTeX users please use one of
%\bibliographystyle{spbasic}      % basic style, author-year citations
%\bibliographystyle{spmpsci}      % mathematics and physical sciences
%\bibliographystyle{spphys}       % APS-like style for physics
%\bibliography{}   % name your BibTeX data base

% Non-BibTeX users please use
\begin{thebibliography}{}
%
% and use \bibitem to create references. Consult the Instructions
% for authors for reference list style.
%
\bibitem{Openstack}
OpenStack Open Source Cloud Computing Software
\url{https://www.openstack.org/}, accessed 2017-01-10

\bibitem{ROCED}
Authors, Article Title, Journal, Volume, , page numbers (year)
=reference for ROCED
\bibitem{packer}

Packer: tool for creating machine and container images for multiple platforms from a single source configuration. 
\url{https://www.packer.io/}, accessed 2017-01-13

\bibitem{puppet}

Puppet Enterprise. ``IT automation for cloud, security, and DevOps.''
\url{https://puppet.com/}, accessed 2017-01-10

\bibitem{SlurmElastic}
Slurm Elastic Computing ....
%\bibitem{RefJ}
% Format for Journal Reference
%Author, Article title, Journal, Volume, page numbers (year)
% Format for books
%\bibitem{RefB}
%Author, Book title, page numbers. Publisher, place (year)
% etc
\end{thebibliography}

\end{document}
% end of file template.tex

